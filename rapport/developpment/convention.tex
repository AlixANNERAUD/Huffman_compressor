\section{Convention}

Afin de garantir une certaines cohérence dans le code, nous avons décidé d'adopter les conventions suivantes :

\subsection{Langage}

\begin{itemize}
    \item Les identifiants dans le code seront écrit en anglais.
    \item Les commentaires seront écrits en français.
    \item Les acronymes sont proscrits.
\end{itemize}


\subsection{Conventions de nommage}

La plupart des conventions de nommage utilisé sont inspirées de celles utilisées par la communauté C.

\begin{itemize}
    \item Les macro et constante (variables constantes, enumerations ...) seront écrites en majuscule et séparées par des tirets du bas.
    \item Les identifiants des types seront écrits en \texttt{CasseDePascal}. Par exemple : \binaryCode.
    \item Les identifiants des variables et des fonctions sont écrits en \texttt{casse\_de\_serpent}. Par exemple : \texttt{binaryFile}.
    \item Dans le cas où les fonctions sont des méthodes (c'est-à-dire qu'elles sont liées à un type), le nom de la fonction sera préfixé par le nom du type. Par exemple : \texttt{binary\_code\_get\_length}.
\end{itemize}

\subsection{Conventions de programmation}

\begin{itemize}
    \item Les variables globales sont interdites, excepté pour les constantes et la gestion des erreurs.
\end{itemize}