\chapter{Conclusion}

\section{Résultats}

Au final, notre compresseur d'Huffman semble performer correctement dans une grande majorité des cas.
En effet, nous avons pu constater que la compression est efficace pour des fichiers d'une certaines taille (supérieur à la taille de la table de statistiques), d'autant plus avec une quantité réduite de caractères différents.
Voici une liste de résultat obtenus avec notre programme :

\begin{table}[h]
\centering
\begin{tabular}{|c|c|c|c|}
\hline
Fichier & Taille initiale & Taille compressée & Rapport \\ \hline
Vide & 0 octets & 2.060 octets & - \\
Fichier de 1 caractère & 1 octet & 2.060 octets & -2060 \% \\ 
Fichier d'un unique caractère & 10.000.000 octets & 2.060 octets & 99,98 \% \\ 
Fichier "lorem ipsum" & 5.553.066 octets & 2.794.592 octets & 49,67 \% \\ 
Fichier totalement aléatoire & 10.000.000 octets & 10.002.060 octets & -0,02 \% \\ 
Fichier équilibré & 10.000.000 octets & 10.002.060 octets & -0,02 \% \\
Fichier déséquilibré & 5.011.517 octets & 4.859.920 octets & 3,02 \% \\
\hline
\end{tabular}
\caption{Résultats de la compression de fichiers sources C.\\
(voir \texttt{programme/tests/functionnalTests.py})}
\end{table}

\section{Bilan}

Ainsi ce projet nous a permis de mettre en pratique les notions vues en cours d'algorithmique avancée et de programmation en C.
En effet, nous avons pu implémenter un algorithme de compression de données, et nous avons pu nous rendre compte des difficultés liées à la programmation en C.
Nous avons pu également nous habituer aux outils usuels du développement logiciel (compilateur, débogueur, gestionnaire de version, etc.).
